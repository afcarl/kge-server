% Generated by Sphinx.
\def\sphinxdocclass{report}
\newif\ifsphinxKeepOldNames \sphinxKeepOldNamestrue
\documentclass[letterpaper,10pt,english]{sphinxmanual}
\usepackage{iftex}

\ifPDFTeX
  \usepackage[utf8]{inputenc}
\fi
\ifdefined\DeclareUnicodeCharacter
  \DeclareUnicodeCharacter{00A0}{\nobreakspace}
\fi
\usepackage{cmap}
\usepackage[T1]{fontenc}
\usepackage{amsmath,amssymb,amstext}
\usepackage{babel}
\usepackage{times}
\usepackage[Bjarne]{fncychap}
\usepackage{longtable}
\usepackage{sphinx}
\usepackage{multirow}
\usepackage{eqparbox}


\addto\captionsenglish{\renewcommand{\figurename}{Fig.\@ }}
\addto\captionsenglish{\renewcommand{\tablename}{Table }}
\SetupFloatingEnvironment{literal-block}{name=Listing }

\addto\extrasenglish{\def\pageautorefname{page}}

\setcounter{tocdepth}{1}


\title{Practicum Documentation}
\date{Sep 15, 2016}
\release{1}
\author{Víctor Fernández Rico}
\newcommand{\sphinxlogo}{}
\renewcommand{\releasename}{Release}
\makeindex

\makeatletter
\def\PYG@reset{\let\PYG@it=\relax \let\PYG@bf=\relax%
    \let\PYG@ul=\relax \let\PYG@tc=\relax%
    \let\PYG@bc=\relax \let\PYG@ff=\relax}
\def\PYG@tok#1{\csname PYG@tok@#1\endcsname}
\def\PYG@toks#1+{\ifx\relax#1\empty\else%
    \PYG@tok{#1}\expandafter\PYG@toks\fi}
\def\PYG@do#1{\PYG@bc{\PYG@tc{\PYG@ul{%
    \PYG@it{\PYG@bf{\PYG@ff{#1}}}}}}}
\def\PYG#1#2{\PYG@reset\PYG@toks#1+\relax+\PYG@do{#2}}

\expandafter\def\csname PYG@tok@sr\endcsname{\def\PYG@tc##1{\textcolor[rgb]{0.14,0.33,0.53}{##1}}}
\expandafter\def\csname PYG@tok@ch\endcsname{\let\PYG@it=\textit\def\PYG@tc##1{\textcolor[rgb]{0.25,0.50,0.56}{##1}}}
\expandafter\def\csname PYG@tok@ni\endcsname{\let\PYG@bf=\textbf\def\PYG@tc##1{\textcolor[rgb]{0.84,0.33,0.22}{##1}}}
\expandafter\def\csname PYG@tok@ow\endcsname{\let\PYG@bf=\textbf\def\PYG@tc##1{\textcolor[rgb]{0.00,0.44,0.13}{##1}}}
\expandafter\def\csname PYG@tok@nb\endcsname{\def\PYG@tc##1{\textcolor[rgb]{0.00,0.44,0.13}{##1}}}
\expandafter\def\csname PYG@tok@c\endcsname{\let\PYG@it=\textit\def\PYG@tc##1{\textcolor[rgb]{0.25,0.50,0.56}{##1}}}
\expandafter\def\csname PYG@tok@sd\endcsname{\let\PYG@it=\textit\def\PYG@tc##1{\textcolor[rgb]{0.25,0.44,0.63}{##1}}}
\expandafter\def\csname PYG@tok@mf\endcsname{\def\PYG@tc##1{\textcolor[rgb]{0.13,0.50,0.31}{##1}}}
\expandafter\def\csname PYG@tok@no\endcsname{\def\PYG@tc##1{\textcolor[rgb]{0.38,0.68,0.84}{##1}}}
\expandafter\def\csname PYG@tok@nc\endcsname{\let\PYG@bf=\textbf\def\PYG@tc##1{\textcolor[rgb]{0.05,0.52,0.71}{##1}}}
\expandafter\def\csname PYG@tok@cs\endcsname{\def\PYG@tc##1{\textcolor[rgb]{0.25,0.50,0.56}{##1}}\def\PYG@bc##1{\setlength{\fboxsep}{0pt}\colorbox[rgb]{1.00,0.94,0.94}{\strut ##1}}}
\expandafter\def\csname PYG@tok@kd\endcsname{\let\PYG@bf=\textbf\def\PYG@tc##1{\textcolor[rgb]{0.00,0.44,0.13}{##1}}}
\expandafter\def\csname PYG@tok@gs\endcsname{\let\PYG@bf=\textbf}
\expandafter\def\csname PYG@tok@nd\endcsname{\let\PYG@bf=\textbf\def\PYG@tc##1{\textcolor[rgb]{0.33,0.33,0.33}{##1}}}
\expandafter\def\csname PYG@tok@vc\endcsname{\def\PYG@tc##1{\textcolor[rgb]{0.73,0.38,0.84}{##1}}}
\expandafter\def\csname PYG@tok@cp\endcsname{\def\PYG@tc##1{\textcolor[rgb]{0.00,0.44,0.13}{##1}}}
\expandafter\def\csname PYG@tok@vg\endcsname{\def\PYG@tc##1{\textcolor[rgb]{0.73,0.38,0.84}{##1}}}
\expandafter\def\csname PYG@tok@cpf\endcsname{\let\PYG@it=\textit\def\PYG@tc##1{\textcolor[rgb]{0.25,0.50,0.56}{##1}}}
\expandafter\def\csname PYG@tok@c1\endcsname{\let\PYG@it=\textit\def\PYG@tc##1{\textcolor[rgb]{0.25,0.50,0.56}{##1}}}
\expandafter\def\csname PYG@tok@ss\endcsname{\def\PYG@tc##1{\textcolor[rgb]{0.32,0.47,0.09}{##1}}}
\expandafter\def\csname PYG@tok@gd\endcsname{\def\PYG@tc##1{\textcolor[rgb]{0.63,0.00,0.00}{##1}}}
\expandafter\def\csname PYG@tok@si\endcsname{\let\PYG@it=\textit\def\PYG@tc##1{\textcolor[rgb]{0.44,0.63,0.82}{##1}}}
\expandafter\def\csname PYG@tok@kn\endcsname{\let\PYG@bf=\textbf\def\PYG@tc##1{\textcolor[rgb]{0.00,0.44,0.13}{##1}}}
\expandafter\def\csname PYG@tok@sc\endcsname{\def\PYG@tc##1{\textcolor[rgb]{0.25,0.44,0.63}{##1}}}
\expandafter\def\csname PYG@tok@vi\endcsname{\def\PYG@tc##1{\textcolor[rgb]{0.73,0.38,0.84}{##1}}}
\expandafter\def\csname PYG@tok@nf\endcsname{\def\PYG@tc##1{\textcolor[rgb]{0.02,0.16,0.49}{##1}}}
\expandafter\def\csname PYG@tok@il\endcsname{\def\PYG@tc##1{\textcolor[rgb]{0.13,0.50,0.31}{##1}}}
\expandafter\def\csname PYG@tok@na\endcsname{\def\PYG@tc##1{\textcolor[rgb]{0.25,0.44,0.63}{##1}}}
\expandafter\def\csname PYG@tok@o\endcsname{\def\PYG@tc##1{\textcolor[rgb]{0.40,0.40,0.40}{##1}}}
\expandafter\def\csname PYG@tok@k\endcsname{\let\PYG@bf=\textbf\def\PYG@tc##1{\textcolor[rgb]{0.00,0.44,0.13}{##1}}}
\expandafter\def\csname PYG@tok@s\endcsname{\def\PYG@tc##1{\textcolor[rgb]{0.25,0.44,0.63}{##1}}}
\expandafter\def\csname PYG@tok@ge\endcsname{\let\PYG@it=\textit}
\expandafter\def\csname PYG@tok@err\endcsname{\def\PYG@bc##1{\setlength{\fboxsep}{0pt}\fcolorbox[rgb]{1.00,0.00,0.00}{1,1,1}{\strut ##1}}}
\expandafter\def\csname PYG@tok@m\endcsname{\def\PYG@tc##1{\textcolor[rgb]{0.13,0.50,0.31}{##1}}}
\expandafter\def\csname PYG@tok@gh\endcsname{\let\PYG@bf=\textbf\def\PYG@tc##1{\textcolor[rgb]{0.00,0.00,0.50}{##1}}}
\expandafter\def\csname PYG@tok@s1\endcsname{\def\PYG@tc##1{\textcolor[rgb]{0.25,0.44,0.63}{##1}}}
\expandafter\def\csname PYG@tok@gu\endcsname{\let\PYG@bf=\textbf\def\PYG@tc##1{\textcolor[rgb]{0.50,0.00,0.50}{##1}}}
\expandafter\def\csname PYG@tok@nv\endcsname{\def\PYG@tc##1{\textcolor[rgb]{0.73,0.38,0.84}{##1}}}
\expandafter\def\csname PYG@tok@go\endcsname{\def\PYG@tc##1{\textcolor[rgb]{0.20,0.20,0.20}{##1}}}
\expandafter\def\csname PYG@tok@w\endcsname{\def\PYG@tc##1{\textcolor[rgb]{0.73,0.73,0.73}{##1}}}
\expandafter\def\csname PYG@tok@gr\endcsname{\def\PYG@tc##1{\textcolor[rgb]{1.00,0.00,0.00}{##1}}}
\expandafter\def\csname PYG@tok@mh\endcsname{\def\PYG@tc##1{\textcolor[rgb]{0.13,0.50,0.31}{##1}}}
\expandafter\def\csname PYG@tok@mb\endcsname{\def\PYG@tc##1{\textcolor[rgb]{0.13,0.50,0.31}{##1}}}
\expandafter\def\csname PYG@tok@kt\endcsname{\def\PYG@tc##1{\textcolor[rgb]{0.56,0.13,0.00}{##1}}}
\expandafter\def\csname PYG@tok@kc\endcsname{\let\PYG@bf=\textbf\def\PYG@tc##1{\textcolor[rgb]{0.00,0.44,0.13}{##1}}}
\expandafter\def\csname PYG@tok@s2\endcsname{\def\PYG@tc##1{\textcolor[rgb]{0.25,0.44,0.63}{##1}}}
\expandafter\def\csname PYG@tok@nn\endcsname{\let\PYG@bf=\textbf\def\PYG@tc##1{\textcolor[rgb]{0.05,0.52,0.71}{##1}}}
\expandafter\def\csname PYG@tok@kr\endcsname{\let\PYG@bf=\textbf\def\PYG@tc##1{\textcolor[rgb]{0.00,0.44,0.13}{##1}}}
\expandafter\def\csname PYG@tok@kp\endcsname{\def\PYG@tc##1{\textcolor[rgb]{0.00,0.44,0.13}{##1}}}
\expandafter\def\csname PYG@tok@se\endcsname{\let\PYG@bf=\textbf\def\PYG@tc##1{\textcolor[rgb]{0.25,0.44,0.63}{##1}}}
\expandafter\def\csname PYG@tok@mi\endcsname{\def\PYG@tc##1{\textcolor[rgb]{0.13,0.50,0.31}{##1}}}
\expandafter\def\csname PYG@tok@bp\endcsname{\def\PYG@tc##1{\textcolor[rgb]{0.00,0.44,0.13}{##1}}}
\expandafter\def\csname PYG@tok@sb\endcsname{\def\PYG@tc##1{\textcolor[rgb]{0.25,0.44,0.63}{##1}}}
\expandafter\def\csname PYG@tok@sx\endcsname{\def\PYG@tc##1{\textcolor[rgb]{0.78,0.36,0.04}{##1}}}
\expandafter\def\csname PYG@tok@cm\endcsname{\let\PYG@it=\textit\def\PYG@tc##1{\textcolor[rgb]{0.25,0.50,0.56}{##1}}}
\expandafter\def\csname PYG@tok@sh\endcsname{\def\PYG@tc##1{\textcolor[rgb]{0.25,0.44,0.63}{##1}}}
\expandafter\def\csname PYG@tok@ne\endcsname{\def\PYG@tc##1{\textcolor[rgb]{0.00,0.44,0.13}{##1}}}
\expandafter\def\csname PYG@tok@mo\endcsname{\def\PYG@tc##1{\textcolor[rgb]{0.13,0.50,0.31}{##1}}}
\expandafter\def\csname PYG@tok@nt\endcsname{\let\PYG@bf=\textbf\def\PYG@tc##1{\textcolor[rgb]{0.02,0.16,0.45}{##1}}}
\expandafter\def\csname PYG@tok@gp\endcsname{\let\PYG@bf=\textbf\def\PYG@tc##1{\textcolor[rgb]{0.78,0.36,0.04}{##1}}}
\expandafter\def\csname PYG@tok@gi\endcsname{\def\PYG@tc##1{\textcolor[rgb]{0.00,0.63,0.00}{##1}}}
\expandafter\def\csname PYG@tok@nl\endcsname{\let\PYG@bf=\textbf\def\PYG@tc##1{\textcolor[rgb]{0.00,0.13,0.44}{##1}}}
\expandafter\def\csname PYG@tok@gt\endcsname{\def\PYG@tc##1{\textcolor[rgb]{0.00,0.27,0.87}{##1}}}

\def\PYGZbs{\char`\\}
\def\PYGZus{\char`\_}
\def\PYGZob{\char`\{}
\def\PYGZcb{\char`\}}
\def\PYGZca{\char`\^}
\def\PYGZam{\char`\&}
\def\PYGZlt{\char`\<}
\def\PYGZgt{\char`\>}
\def\PYGZsh{\char`\#}
\def\PYGZpc{\char`\%}
\def\PYGZdl{\char`\$}
\def\PYGZhy{\char`\-}
\def\PYGZsq{\char`\'}
\def\PYGZdq{\char`\"}
\def\PYGZti{\char`\~}
% for compatibility with earlier versions
\def\PYGZat{@}
\def\PYGZlb{[}
\def\PYGZrb{]}
\makeatother

\renewcommand\PYGZsq{\textquotesingle}

\begin{document}

\maketitle
\tableofcontents
\phantomsection\label{index::doc}


Contents:
\phantomsection\label{index:module-dataset}\index{dataset (module)}\index{Dataset (class in dataset)}

\begin{fulllineitems}
\phantomsection\label{index:dataset.Dataset}\pysiglinewithargsret{\sphinxstrong{class }\sphinxcode{dataset.}\sphinxbfcode{Dataset}}{\emph{new\_endpoint=None}, \emph{thread\_limiter=100}}{}~\index{add\_element() (dataset.Dataset method)}

\begin{fulllineitems}
\phantomsection\label{index:dataset.Dataset.add_element}\pysiglinewithargsret{\sphinxbfcode{add\_element}}{\emph{element}, \emph{complete\_list}, \emph{complete\_list\_dict}, \emph{only\_uri=False}}{}
Add element to a list of the dataset. Avoids duplicate elements.
\begin{quote}\begin{description}
\item[{Parameters}] \leavevmode\begin{itemize}
\item {} 
\textbf{\texttt{element}} (\emph{\texttt{str}}) -- The element that will be added to list

\item {} 
\textbf{\texttt{complete\_list}} (\emph{\texttt{list}}) -- The list in which will be added

\item {} 
\textbf{\texttt{complete\_list\_dict}} (\emph{\texttt{dict}}) -- The dict which represents the list.

\item {} 
\textbf{\texttt{only\_uri}} (\emph{\texttt{bool}}) -- Allow load objects distincts than URI's

\end{itemize}

\item[{Returns}] \leavevmode
The id on the list of the added element

\item[{Return type}] \leavevmode
int

\end{description}\end{quote}

\end{fulllineitems}

\index{build\_levels() (dataset.Dataset method)}

\begin{fulllineitems}
\phantomsection\label{index:dataset.Dataset.build_levels}\pysiglinewithargsret{\sphinxbfcode{build\_levels}}{\emph{n\_levels}}{}
Generates a simple \emph{chain} of triplets for the desired levels
\begin{quote}\begin{description}
\item[{Parameters}] \leavevmode
\textbf{\texttt{n\_levels}} (\emph{\texttt{int}}) -- Deep of the search on wikidata graph

\item[{Returns}] \leavevmode
A list of chained triplets

\item[{Return type}] \leavevmode
list

\end{description}\end{quote}

\end{fulllineitems}

\index{build\_n\_levels\_query() (dataset.Dataset method)}

\begin{fulllineitems}
\phantomsection\label{index:dataset.Dataset.build_n_levels_query}\pysiglinewithargsret{\sphinxbfcode{build\_n\_levels\_query}}{\emph{n\_levels=3}}{}
Builds a CONSTRUCT SPARQL query of the desired deep
\begin{quote}\begin{description}
\item[{Parameters}] \leavevmode
\textbf{\texttt{n\_levels}} (\emph{\texttt{int}}) -- Deep of the search on wikidata graph

\item[{Returns}] \leavevmode
The desired chained query

\item[{Return type}] \leavevmode
string

\end{description}\end{quote}

\end{fulllineitems}

\index{execute\_query() (dataset.Dataset method)}

\begin{fulllineitems}
\phantomsection\label{index:dataset.Dataset.execute_query}\pysiglinewithargsret{\sphinxbfcode{execute\_query}}{\emph{query}, \emph{headers=\{`Accept': `application/json'\}}}{}
Executes a SPARQL query to the endpoint
\begin{quote}\begin{description}
\item[{Parameters}] \leavevmode
\textbf{\texttt{query}} (\emph{\texttt{str}}) -- The SPARQL query

\item[{Returns}] \leavevmode
A tuple compound of (http\_status, json\_or\_error)

\end{description}\end{quote}

\end{fulllineitems}

\index{exist\_element() (dataset.Dataset method)}

\begin{fulllineitems}
\phantomsection\label{index:dataset.Dataset.exist_element}\pysiglinewithargsret{\sphinxbfcode{exist\_element}}{\emph{element}, \emph{complete\_list\_dict}}{}
Check if element exists on a given list
\begin{quote}\begin{description}
\item[{Parameters}] \leavevmode\begin{itemize}
\item {} 
\textbf{\texttt{element}} (\emph{\texttt{str}}) -- The element itself

\item {} 
\textbf{\texttt{complete\_list\_dict}} (\emph{\texttt{dict}}) -- The dictionary to search in

\end{itemize}

\item[{Returns}] \leavevmode
Wether the item was found or no

\item[{Return type}] \leavevmode
bool

\end{description}\end{quote}

\end{fulllineitems}

\index{extract\_entity() (dataset.Dataset method)}

\begin{fulllineitems}
\phantomsection\label{index:dataset.Dataset.extract_entity}\pysiglinewithargsret{\sphinxbfcode{extract\_entity}}{\emph{entity}, \emph{filters=\{`wdt-entity': True}, \emph{`wdt-prop': True}, \emph{`bnode': False}, \emph{`wdt-reference': False}, \emph{`wdt-statement': False}, \emph{`literal': False\}}}{}
Given an entity, returns the valid representation, ready to be saved

The filter argument allows to avoid adding elements into lists that
will not be used. It is a dictionary with the shape: \{`filter': bool\}.
The valid filters (and default) are:
\begin{itemize}
\item {} 
\emph{wdt-entity} - True

\item {} 
\emph{wdt-reference} - False

\item {} 
\emph{wdt-statement} - True

\item {} 
\emph{wdt-prop} - True

\item {} 
\emph{literal} - False

\item {} 
\emph{bnode} - False

\end{itemize}
\begin{quote}\begin{description}
\item[{Parameters}] \leavevmode\begin{itemize}
\item {} 
\textbf{\texttt{entity}} (\emph{\texttt{dict}}) -- The entity to be analyzed

\item {} 
\textbf{\texttt{filters}} (\emph{\texttt{dict}}) -- A dictionary to allow filter entities

\end{itemize}

\item[{Returns}] \leavevmode
The entity itself or False

\end{description}\end{quote}

\end{fulllineitems}

\index{load\_dataset\_from\_json() (dataset.Dataset method)}

\begin{fulllineitems}
\phantomsection\label{index:dataset.Dataset.load_dataset_from_json}\pysiglinewithargsret{\sphinxbfcode{load\_dataset\_from\_json}}{\emph{json}, \emph{only\_uri=False}}{}
Loads the dataset object with a JSON

The JSON structure required is:
\{`object': \{\}, `subject': \{\}, `predicate': \{\}\}
\begin{quote}\begin{description}
\item[{Parameters}] \leavevmode\begin{itemize}
\item {} 
\textbf{\texttt{json}} (\emph{\texttt{list}}) -- A list of dictionary parsed from JSON

\item {} 
\textbf{\texttt{only\_uri}} (\emph{\texttt{bool}}) -- Allow load objects distincts than URI's

\end{itemize}

\end{description}\end{quote}

\end{fulllineitems}

\index{load\_dataset\_from\_nlevels() (dataset.Dataset method)}

\begin{fulllineitems}
\phantomsection\label{index:dataset.Dataset.load_dataset_from_nlevels}\pysiglinewithargsret{\sphinxbfcode{load\_dataset\_from\_nlevels}}{\emph{nlevels}, \emph{extra\_params='`}, \emph{only\_uri=False}}{}
Builds a nlevels query, executes, and loads data on object
\begin{quote}\begin{description}
\item[{Deprecated}] \leavevmode
\item[{Parameters}] \leavevmode\begin{itemize}
\item {} 
\textbf{\texttt{nlevels}} (\emph{\texttt{int}}) -- Deep of the search on wikidata graph

\item {} 
\textbf{\texttt{extra\_params}} (\emph{\texttt{str}}) -- Extra SPARQL instructions for the query

\item {} 
\textbf{\texttt{only\_uri}} (\emph{\texttt{bool}}) -- Allow load objects distincts than URI's

\end{itemize}

\end{description}\end{quote}

\end{fulllineitems}

\index{load\_dataset\_from\_query() (dataset.Dataset method)}

\begin{fulllineitems}
\phantomsection\label{index:dataset.Dataset.load_dataset_from_query}\pysiglinewithargsret{\sphinxbfcode{load\_dataset\_from\_query}}{\emph{query}, \emph{only\_uri=False}}{}
Receives a Sparql query and fills dataset object with the response

The method will execute the query itself and will call to other method
to fill in the dataset object
\begin{quote}\begin{description}
\item[{Parameters}] \leavevmode\begin{itemize}
\item {} 
\textbf{\texttt{query}} (\emph{\texttt{str}}) -- A valid SPARQL query

\item {} 
\textbf{\texttt{only\_uri}} (\emph{\texttt{bool}}) -- Allow load objects distincts than URI's

\end{itemize}

\end{description}\end{quote}

\end{fulllineitems}

\index{load\_dataset\_recurrently() (dataset.Dataset method)}

\begin{fulllineitems}
\phantomsection\label{index:dataset.Dataset.load_dataset_recurrently}\pysiglinewithargsret{\sphinxbfcode{load\_dataset\_recurrently}}{\emph{levels}, \emph{verbose=1}}{}
Loads to dataset all entities with BNE ID and their relations

Due to Wikidata endpoint cann't execute queries that take long time
to complete, it is necessary to consruct the dataset entity by entity,
without using SPARQL CONSTRUCT. This method will start concurrently
some threads to make several SPARQL SELECT queries.
\begin{quote}\begin{description}
\item[{Parameters}] \leavevmode\begin{itemize}
\item {} 
\textbf{\texttt{levels}} (\emph{\texttt{int}}) -- The depth to get triplets related with original item

\item {} 
\textbf{\texttt{verbose}} (\emph{\texttt{int}}) -- The level of verbosity. 0 is low, and 2 is high

\end{itemize}

\item[{Returns}] \leavevmode
True if operation was successful

\item[{Return type}] \leavevmode
bool

\end{description}\end{quote}

\end{fulllineitems}

\index{load\_entire\_dataset() (dataset.Dataset method)}

\begin{fulllineitems}
\phantomsection\label{index:dataset.Dataset.load_entire_dataset}\pysiglinewithargsret{\sphinxbfcode{load\_entire\_dataset}}{\emph{levels}, \emph{where='`}, \emph{batch=100000}, \emph{verbose=True}}{}
Loads the dataset by quering to Wikidata on the desired levels
\begin{quote}\begin{description}
\item[{Deprecated}] \leavevmode
\item[{Parameters}] \leavevmode\begin{itemize}
\item {} 
\textbf{\texttt{levels}} (\emph{\texttt{int}}) -- Deep of the search

\item {} 
\textbf{\texttt{where}} (\emph{\texttt{str}}) -- Extra where statements for SPARQL query

\item {} 
\textbf{\texttt{batch}} (\emph{\texttt{int}}) -- Number of elements returned each query

\item {} 
\textbf{\texttt{verbose}} (\emph{\texttt{bool}}) -- True for showing all steps the method do

\end{itemize}

\item[{Returns}] \leavevmode
True if operation was successful

\item[{Return type}] \leavevmode
bool

\end{description}\end{quote}

\end{fulllineitems}

\index{load\_from\_binary() (dataset.Dataset method)}

\begin{fulllineitems}
\phantomsection\label{index:dataset.Dataset.load_from_binary}\pysiglinewithargsret{\sphinxbfcode{load\_from\_binary}}{\emph{filepath}}{}
Loads the dataset object from the disk

Loads this dataset object with the binary file
\begin{quote}\begin{description}
\item[{Parameters}] \leavevmode
\textbf{\texttt{filepath}} (\emph{\texttt{str}}) -- The path of the binary file

\item[{Returns}] \leavevmode
True if operation was successful

\item[{Return type}] \leavevmode
bool

\end{description}\end{quote}

\end{fulllineitems}

\index{save\_to\_binary() (dataset.Dataset method)}

\begin{fulllineitems}
\phantomsection\label{index:dataset.Dataset.save_to_binary}\pysiglinewithargsret{\sphinxbfcode{save\_to\_binary}}{\emph{filepath}}{}
Saves the dataset object on the disk

The dataset will be saved with the required format for reading
from the original library, and is prepared to be trained.
\begin{quote}\begin{description}
\item[{Parameters}] \leavevmode
\textbf{\texttt{filepath}} (\emph{\texttt{str}}) -- The path of the file where should be saved

\item[{Returns}] \leavevmode
True if operation was successful

\item[{Return type}] \leavevmode
bool

\end{description}\end{quote}

\end{fulllineitems}

\index{show() (dataset.Dataset method)}

\begin{fulllineitems}
\phantomsection\label{index:dataset.Dataset.show}\pysiglinewithargsret{\sphinxbfcode{show}}{\emph{verbose=False}}{}
Show all elements of the dataset

By default prints only one line with the number of entities, relations
and triplets. If verbose, prints every list. Use wisely
\begin{quote}\begin{description}
\item[{Parameters}] \leavevmode
\textbf{\texttt{verbose}} (\emph{\texttt{bool}}) -- If true prints every item of all lists

\end{description}\end{quote}

\end{fulllineitems}

\index{train\_split() (dataset.Dataset method)}

\begin{fulllineitems}
\phantomsection\label{index:dataset.Dataset.train_split}\pysiglinewithargsret{\sphinxbfcode{train\_split}}{\emph{ratio=0.8}}{}
Split subs into three lists: train, valid and test

The triplets should have a specific name and size to be compatible
with the original library. Splits the original triplets (self.subs) in
three different lists: \emph{train\_subs}, \emph{valid\_subs} and \emph{test\_subs}.
The `ratio' param will leave that quantity for train\_subs, and the
rest will be a half for valid and the other half for test
\begin{quote}\begin{description}
\item[{Parameters}] \leavevmode
\textbf{\texttt{ratio}} (\emph{\texttt{float}}) -- The ratio of all triplets required for \emph{train\_subs}

\item[{Returns}] \leavevmode
A dictionary with splited subs

\item[{Return type}] \leavevmode
dict

\end{description}\end{quote}

\end{fulllineitems}


\end{fulllineitems}



\chapter{Indices and tables}
\label{index:indices-and-tables}\label{index:welcome-to-practicum-s-documentation}\begin{itemize}
\item {} 
\DUrole{xref,std,std-ref}{genindex}

\item {} 
\DUrole{xref,std,std-ref}{modindex}

\item {} 
\DUrole{xref,std,std-ref}{search}

\end{itemize}


\renewcommand{\indexname}{Python Module Index}
\begin{theindex}
\def\bigletter#1{{\Large\sffamily#1}\nopagebreak\vspace{1mm}}
\bigletter{d}
\item {\texttt{dataset}}, \pageref{index:module-dataset}
\end{theindex}

\renewcommand{\indexname}{Index}
\printindex
\end{document}
